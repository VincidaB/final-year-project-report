\newglossaryentry{ordonnancement}{%
    name = {ordonnancement},
    description = { ou \textit{scheduling} est l'activité qui consiste à affecter des ressources à des tâches}
}
\newglossaryentry{ordonnanceur}{%
    name = {ordonnanceur},
    description = { ou \textit{scheduler} est un composant d'un système d'exploitation chargé de gérer l'ordonnancement des processus}
}
\newglossaryentry{processeur}{%
    name = {processeur},
    description = { ou \textit{CPU} est un composant présent dans tout ordinateur. Il est chargé d'effectuer les calculs et de gérer les flux de données}
}
\newglossaryentry{plateforme heterogene}{%
    name = {plateforme hétérogène},
    description = {système formé d'un ensemble de processeurs différents}
}
\newglossaryentry{SoC}{%
    name = {SoC},
    description = { ou \textit{Système On a Chip} est un circuit intégré qui rassemble sur une même puce plusieurs composants d'un ordinateur}
}

\newglossaryentry{cluster}{%
    name = {cluster},
    description = {ensemble interconnecté de plusieurs processeurs}
}

\newglossaryentry{bootloader}{%
    name = {bootloader},
    description = {court programme chargé au démarrage de l'ordinateur initialisant le système d'exploitation}
}
\newglossaryentry{git}{%
    name = {git},
    description = {système de gestion de versions décentralisé, utilisé pour suivre les modifications apportées à des fichiers sources dans un projet de développement logiciel}
}
\newglossaryentry{preemption}{%
    name = {préemption},
    plural = {préemptions},
    description = {processus par lequel un système d'exploitation interrompt temporairement l'exécution d'une tâche en cours pour donner la priorité à une autre tâche de plus haute priorité}
}
\newglossaryentry{SHRIMP}{%
    name = {SHRIMP},
    description = {\textit{Scheduling of Real-Time Heterogeneous Multiprocessor Platform} ou Ordonnancement Temps réel de Plateforme Multiprocesseur Hétérogène}
}
\newglossaryentry{checksum md5}{%
    name = {checksum md5},
    description = {algorithme de hachage cryptographique de 128 bits permettant de produire un résultat (appelé aussi empreinte) à partir d'un fichier}
}